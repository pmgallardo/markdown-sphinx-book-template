%% Generated by Sphinx.
\def\sphinxdocclass{report}
\documentclass[a4paper,11pt,english]{sphinxmanual}
\ifdefined\pdfpxdimen
   \let\sphinxpxdimen\pdfpxdimen\else\newdimen\sphinxpxdimen
\fi \sphinxpxdimen=.75bp\relax
\ifdefined\pdfimageresolution
    \pdfimageresolution= \numexpr \dimexpr1in\relax/\sphinxpxdimen\relax
\fi
\newdimen\sphinxremdimen\sphinxremdimen = 11pt
%% let collapsible pdf bookmarks panel have high depth per default
\PassOptionsToPackage{bookmarksdepth=5}{hyperref}
%% turn off hyperref patch of \index as sphinx.xdy xindy module takes care of
%% suitable \hyperpage mark-up, working around hyperref-xindy incompatibility
\PassOptionsToPackage{hyperindex=false}{hyperref}
%% memoir class requires extra handling
\makeatletter\@ifclassloaded{memoir}
{\ifdefined\memhyperindexfalse\memhyperindexfalse\fi}{}\makeatother

\PassOptionsToPackage{booktabs}{sphinx}
\PassOptionsToPackage{colorrows}{sphinx}

\PassOptionsToPackage{warn}{textcomp}

\catcode`^^^^00a0\active\protected\def^^^^00a0{\leavevmode\nobreak\ }
\usepackage{cmap}
\usepackage{fontspec}
\defaultfontfeatures[\rmfamily,\sffamily,\ttfamily]{}
\usepackage{amsmath,amssymb,amstext}
\usepackage{polyglossia}
\setmainlanguage{english}



\setmainfont{FreeSerif}[
  Extension      = .otf,
  UprightFont    = *,
  ItalicFont     = *Italic,
  BoldFont       = *Bold,
  BoldItalicFont = *BoldItalic
]
\setsansfont{FreeSans}[
  Extension      = .otf,
  UprightFont    = *,
  ItalicFont     = *Oblique,
  BoldFont       = *Bold,
  BoldItalicFont = *BoldOblique,
]
\setmonofont{FreeMono}[Scale=0.9,
  Extension      = .otf,
  UprightFont    = *,
  ItalicFont     = *Oblique,
  BoldFont       = *Bold,
  BoldItalicFont = *BoldOblique,
]



\usepackage[Bjarne]{fncychap}
\usepackage{sphinx}

\fvset{fontsize=auto}
\usepackage{geometry}


% Include hyperref last.
\usepackage{hyperref}
% Fix anchor placement for figures with captions.
\usepackage{hypcap}% it must be loaded after hyperref.
% Set up styles of URL: it should be placed after hyperref.
\urlstyle{same}

\addto\captionsenglish{\renewcommand{\contentsname}{Front Matter}}

\usepackage{sphinxmessages}
\setcounter{tocdepth}{1}



\title{My Book}
\date{Jan 29, 2026}
\release{1.0}
\author{Pablo Gallardo}
\newcommand{\sphinxlogo}{\vbox{}}
\renewcommand{\releasename}{Release}
\makeindex
\begin{document}

\pagestyle{empty}
\sphinxmaketitle
\pagestyle{plain}
\sphinxtableofcontents
\pagestyle{normal}
\phantomsection\label{\detokenize{index::doc}}


\sphinxstepscope


\chapter{My Technical Book}
\label{\detokenize{cover:my-technical-book}}\label{\detokenize{cover::doc}}

\section{A Practical Guide to Building Things}
\label{\detokenize{cover:a-practical-guide-to-building-things}}
\sphinxAtStartPar
\sphinxstylestrong{Author:} Your Name\\
\sphinxstylestrong{Version:} 1.0

\sphinxAtStartPar
Welcome to this technical book. This page serves as the cover.

\sphinxstepscope


\chapter{Chapter 1 — Foundations}
\label{\detokenize{chapter1:chapter-1-foundations}}\label{\detokenize{chapter1::doc}}
\sphinxAtStartPar
This chapter introduces core concepts used throughout the book.


\section{Key term: Widget Architecture}
\label{\detokenize{chapter1:key-term-widget-architecture}}\label{\detokenize{chapter1:widget-architecture}}
\sphinxAtStartPar
A \sphinxstylestrong{widget} is a modular unit with a stable interface.

\index{widget@\spxentry{widget}}\ignorespaces 
\sphinxAtStartPar
See the implementation details in {\hyperref[\detokenize{chapter2::doc}]{\sphinxcrossref{\DUrole{doc}{Chapter 2 — Implementation}}}}.

\sphinxstepscope


\chapter{Chapter 2 — Implementation}
\label{\detokenize{chapter2:chapter-2-implementation}}\label{\detokenize{chapter2::doc}}
\sphinxAtStartPar
In {\hyperref[\detokenize{chapter1:widget-architecture}]{\sphinxcrossref{\DUrole{std}{\DUrole{std-ref}{Key term: Widget Architecture}}}}}, we defined widget architecture.


\section{A minimal example}
\label{\detokenize{chapter2:a-minimal-example}}
\begin{sphinxVerbatim}[commandchars=\\\{\}]
\PYG{k}{class}\PYG{+w}{ }\PYG{n+nc}{Widget}\PYG{p}{:}
    \PYG{k}{def}\PYG{+w}{ }\PYG{n+nf+fm}{\PYGZus{}\PYGZus{}init\PYGZus{}\PYGZus{}}\PYG{p}{(}\PYG{n+nb+bp}{self}\PYG{p}{,} \PYG{n}{name}\PYG{p}{:} \PYG{n+nb}{str}\PYG{p}{)}\PYG{p}{:}
        \PYG{n+nb+bp}{self}\PYG{o}{.}\PYG{n}{name} \PYG{o}{=} \PYG{n}{name}

    \PYG{k}{def}\PYG{+w}{ }\PYG{n+nf}{run}\PYG{p}{(}\PYG{n+nb+bp}{self}\PYG{p}{)} \PYG{o}{\PYGZhy{}}\PYG{o}{\PYGZgt{}} \PYG{n+nb}{str}\PYG{p}{:}
        \PYG{k}{return} \PYG{l+s+sa}{f}\PYG{l+s+s2}{\PYGZdq{}}\PYG{l+s+s2}{Widget }\PYG{l+s+si}{\PYGZob{}}\PYG{n+nb+bp}{self}\PYG{o}{.}\PYG{n}{name}\PYG{l+s+si}{\PYGZcb{}}\PYG{l+s+s2}{ running}\PYG{l+s+s2}{\PYGZdq{}}
\end{sphinxVerbatim}

\sphinxstepscope


\chapter{References}
\label{\detokenize{references:references}}\label{\detokenize{references::doc}}

\section{Documentation}
\label{\detokenize{references:documentation}}\begin{itemize}
\item {} 
\sphinxAtStartPar
Python standard library

\item {} 
\sphinxAtStartPar
Sphinx documentation

\end{itemize}

\index{references@\spxentry{references}}\ignorespaces 


\renewcommand{\indexname}{Index}
\printindex
\end{document}